\chapter*{Введение}	%	Пишем "Введение" вместо  "Глава 1 Введение"
\addcontentsline{toc}{chapter}{Введение}	%	Добавляем "Введение" в оглавление
В наше время сложно представить себе человека без сотового телефона, планшетного компьютера, смартфона или любого другого портативного мультимедийного устройства. Мы привыкли к тому, что всегда под рукой не только средство связи, но и множество полезных функций, таких как: калькулятор, органайзер, конвертер, календарь, часы. Смартфоны становятся новой мобильной игровой платформой, соревнуясь с классическими карманными игровыми системами вроде Nintendo DS или Playstation Portable.
В устройстве смартфона все довольно просто. \\*
\indent
Главным образом он состоит из нескольких отдельных блоков - памяти, процессора, который занимается вычислениями, хранилища данных, радиомодуля, который в свою очередь состоит из приемника и передатчика и отвечает за связь. Самое интересное здесь - операционная система, установленная на встроенную память. От операционной системы и ее версии зависят все основные возможности устройства. Смартфоны, как и персональные компьютеры, существуют в абсолютно разных комплектациях и под управлением разных операционных систем, разновидности которых мы рассмотрим далее.
По мере роста продаж мобильных устройств во всем мире, растет и спрос на различные приложения для них. \\
\indent
Каждая уважающая себя компания, стремится иметь хотя бы одно мобильное приложение, чтобы быть у своего клиента "всегда под рукой". А существование некоторых компаний и вовсе сложно представить без мобильных устройств и специализированных программ, при помощи которых можно, например, управлять базами данных или следить за состоянием своего продукта на рынке в любой момент времени.
К сожалению, на сегодняшний день не существуют определенного стандарта средства разработки мобильных приложений. Каждый производитель пытается сделать операционную систему в своем устройстве более уникальной и запоминающейся пользователю, и как следствие возникают вопросы совместимости различных приложений на разных операционных системах.

Основной целью данной работы является разработка клиент-серверного игрового приложения на примере игры жанра Shooter для мобильных устройств на базе операционной системы Android.

Android - портативная (сетевая) операционная система для коммуникаторов, планшетных компьютеров, электронных книг, цифровых проигрывателей, наручных часов и нетбуков основанная на ядре Linux. Изначально разрабатывалась компанией Android Inc., которую затем купила Google. Впоследствии Google инициировала создание альянса Open Handset Alliance (OHA), который сейчас занимается поддержкой и дальнейшим развитием платформы. Android позволяет создавать Java-приложения, управляющие устройством через разработанные Google библиотеки. Android Native Development Kit позволяет портировать (но не отлаживать) библиотеки и компоненты приложений, написанные на С и других языках;

% TODO актуальность

Для достижение поставленной цели необходимо решить следующие задачи:%они мне не до конца нравятся
\begin{itemize}
  \item Создание инфраструктуры сервера

  \item Проектирование базы данных

  \item Разработка и проектирование User Interface
  
  \item Создание логики и описание поведения игровых моделей
  
  \item Разработка дизайна приложения
\end{itemize}

\vspace{3pc}

% и мне совершенно не нравятся как написаны эти два абзаца
В качестве языка разработки для сервера и клиента был выбран язык Java. Мы посчитали его наиболее приемлимым для клиентской части нашего приложения, поскольку наиболее развивающиеся и поддерживаемые средства разработки под ОС Android (Android SDK) основаны на использовании Java. А для максимальной совместимости сервера и клиента для реализации сервера был также выбран этот язык.
